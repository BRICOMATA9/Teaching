\mainlanguage[fr]

\language [fr]

\setuppapersize[A4][A4]
\setupcolors[state=start]
\setuplayout [backspace=1.75cm,topspace=2.475cm,width=16.7825cm,height=23.735cm]
\setuppagenumbering  [alternative=doublesided,location=]
\setupheadertexts[Les fiches à Bébert][\CONTEXT, la mise en page][ ][section]
\setupheader [state=empty]
\setupfootertexts[ ][pagenumber]
%%%%%%%%%%%%%%%%%%%%%%%%%%%%
%%% des boites colorées %%%%
%%%%%%%%%%%%%%%%%%%%%%%%%%%%          
\defineframedtext[Remarque][width=0.7\makeupwidth,location=middle,framecolor=darkgreen,
rulethickness=3pt,corner=round,radius=10pt
]%attention si l'on met corner il faut automatiquement préciser radius (valeur 0 par défaut et erreur de compilation)

\defineframedtext[Alert][width=0.8\makeupwidth,location=middle,framecolor=darkred,
rulethickness=3pt,corner=round,radius=10pt]


\defineframed
[cadreVert]
[corner=round,framecolor=darkgreen]
\defineframed[boiteBleue]
[frame=off,
background=color,
backgroundcolor=blue]
%%%%%%%%%%%%%%%%%%%%%%
%%% hyperlien url %%%%
%%%%%%%%%%%%%%%%%%%%%%         
\setupcolors
  [state=start]
\setupinteraction
  [state=start,color=blue]
\useURL [garden][http://contextgarden.net][][Context garden]
\useURL[pragma][http://www.pragma-ade.com/][][Pragma ADE]
\useURL[gutenberg][http://www.gutenberg.eu.org/spip.php?article110][][GUTenberg le Groupe francophone des Utilisateurs de \TeX, \LaTeX]
\useURL[wikiFrame][http://wiki.contextgarden.net/Reference/en/framed][][ConTeXt command reference]
\useURL[wikiFormaPapier][http://fr.wikipedia.org/wiki/ISO_216][][wikipedia, la norme iso216]
\useURL[wikiPapier][http://fr.wikipedia.org/wiki/Format_de_papier][][wikipedia, le format de papier]

\useURL[empagement][http://www.alain.les-hurtig.org/varia/empagement.html][][L'outil typographique]
\useURL[empagement2][http://cahiers.gutenberg.eu.org/cg-bin/article/CG_2003___42_4_0.pdf][][Étude comparative
de différents modèles d’empagement de Markus {\sc \bf Kohm}]

%%%%%%%%%%%%%%%%%%%%%%
%%% commandes perso  %%%%
%%%%%%%%%%%%%%%%%%%%%% 
\def\Val#1{Valeurs = {\tt #1}\blank[small]} 